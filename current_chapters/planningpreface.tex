In the following chapters, we discuss the \textit{planning} problem where a model is known. Before diving in, however, we shall spend some time on defining the various approaches to modeling a sequential decision problem, and motivate our choice to focus on some of them. In the next chapters, we will rigorously cover selected approaches and their implications. This chapter is quite different from the rest of the book, as it discusses epistemological and philosophical  issues more than anything else. 

We are interested in sequential decision problems in which a sequence of decisions need to be taken in order to achieve a goal or optimize some performance measure. Some examples include:
\begin{example}[Board games]
    An agent playing a board game such as Tic-Tac-Toe, chess, or backgammon. Board games are typically played against an opponent, and may involve external randomness such as the dice in backgammon. The goal is to play a sequence of moves that lead to winning the game.
\end{example}

\begin{example}[Robot Control]
    A robot needs to be controlled to perform some task, for example, picking up an object and placing it in a bin, or folding up a piece of cloth. The robot is controlled by applying voltages to its motors, and the goal is to find a sequence of controls that perform the desired task within some time limits.
\end{example}

\begin{example}[Inventory Control]
Inventory control represents a classical and practical applications of sequential decision making under uncertainty. In its simplest form, a decision maker must determine how much inventory to order at each time period to meet uncertain future demand while balancing ordering costs, holding costs, and stockout penalties. The uncertainty in demand requires a good policy to adapt to the stochastic nature of customer behavior while accounting for both immediate costs and future implications of current decisions.
The $(s,S)$ policy, also known as a reorder point-order-up-to policy (\cite{scarf1959optimality}), is an elegantly simple yet often optimal approach to inventory control. Under this policy, whenever the inventory level drops to or below a reorder point $s$, an order is placed to bring the inventory position up to a target level $S$. While finding the optimal values for $s$ and $S$ is non-trivial, this policy structure has been proven optimal for many important inventory problems under reasonable assumptions. The $(s,S)$ framework provides an excellent example of how constraining the policy space, in this case to just two parameters, can make learning more efficient while still achieving strong performance. \end{example}
%the importance of modeling
When we are given a sequential decision problem we have to model it from a mathematical perspective. In this book, and in much of the literature, the focus is mostly on the celebrated Markov Decision Process (MDP) model. It should be clear that this is merely a model, i.e., one should not view it as a precise reflection of reality. To quote Box is ``{\it all models are wrong, but some are useful}''. Our goal is to have useful models and as such the Markov decision model is a perfect example. The MDP model has the following components which we discuss here and provide formally in later chapters. We will use the agent-centric view, assuming an agent interacts with an environment. This agent is sometimes called a ``decision maker", especially in the operations research community. 
\begin{enumerate}
    \item States: A state is the atomic entity that represents all the information needed to predict future rewards of the system. The agent in an MDP can fully observe the state.
    \item Actions: An action is what can be affected by the decision maker.
    \item Rewards: the rewards represent some numerical measurement that the decision maker wishes to maximize. The reward is assumed to be a function of the current state and the action.
    \item Dynamics: The state changes (or transitions) according to the dynamics. This evolution depends only on the current state and the action chosen but not on future or past states on actions.  
\end{enumerate}


In planning, it is assumed that all the components are known. The objective of the decision maker is to find a policy, i.e., a mapping from histories of state observations to actions, that maximizes some objective function of the reward. We will adopt the following standard assumptions concerning the planning model:
\begin{enumerate}
    \item Time is discrete and regular: decisions are made in some predefined decision epochs. For example, every second/month/year. While continuous time is especially common in robotic applications, we will adhere for simplicity to discrete regular times. In principle, this is not a particularly limiting assumption, as most digital systems inherently discretize the time measurement. However, it may be unnecessary to apply a different control at \textit{every} time step; The semi-MDP model is a common framework to use when the decision epochs are irregular \cite{puterman2014markov}, and there is an extensive literature on optimal control in continuous time \cite{kirk2004optimal}, which we will not consider here. 
    \item Action space is finite. We will mostly assume that the available actions a decision maker can choose from belong to a finite set. While this assumption may appear natural in board games, or any digital system that is discretized, in some domains such as robotics it is more natural to consider a continuous control setting.
    % robotic tasks where control is continuous it does not hold. 
    For continuous actions, the \textit{structure} of the action space is critical for effective decision making -- we will discuss some specific examples, such as a linear dynamical system here. More general continuous and hybrid discrete-continuous models are often studied in the control literature \cite{bemporad1999control} and in operations research \cite{powell2007approximate}.
    \item State space is finite. The set of possible system states is also assumed to be finite and unstructured. The finiteness assumption is mostly a convenience, as any bounded continuous space can be finely discretized to a finite, but very large set. Indeed, in the second part of this book, we shall study learning-based methods that can handle very large state spaces. For problems where the state space has a known and convenient structure, a model that takes this structure into account can be more appropriate. For example, in a linear controlled dynamical system, which we discuss in Section \ref{sec:continuous_control}, the state space is continuous, and its evolution with respect to the control is linear, leading to a closed form optimal solution when the reward has a particular quadratic structure. In the classical STRIPS and PDDL planning models, which we do not cover here, the state space is a list of binary variables (e.g., a system for robot setting a table may be described by [robot gripper closed = False, cup on table = True, plate on table = False,\dots]), and planning algorithms that try to find actions that lead to certain goal states being `true' can take account of this special structure \cite{russell2016artificial}. 
    % With the exception of Chapter \ref{chapter:approx-dp} our basic assumption is that the set of states the system can take is 
    \item Rewards are all given in a single currency. We assume that the agent has a single reward stream it tries to optimize. Specifically the agent tries to maximize the long term sum of rewards. In some cases, a user may be interested in other statistics of the reward, such as its variance~\cite{mannor2013algorithmic}, or to balance multiple types of rewards~\cite{mannor2004geometric}; we do not cover these cases here.
    % discounted reward. 
    \item Markov state evolution. We shall assume that the environment's reaction to the agent's actions is fixed (it may be stochastic, but with a fixed distribution), and depends only on the current state. This assumption precludes environments that are \textit{adversarial} to the agent, or systems with multiple independent agents that learn together with our agent~\cite{Cesa-Bianchi-Lugosi-book,Zhang2021}.
\end{enumerate}

As should be clear from the points above, the MDP model is agnostic to structure that certain problems may posses, and more specialized models may exploit. The reader may question, therefore, why study such a model for planning. As it turns out, the simplicity and generality of the MDP is actually a blessing when using it for \textit{learning}, which is the main focus of this book. The reason is that structure of a specific problem may be implicitly picked up by the learning algorithm, which is designed to identify patterns in data. This strategy has proved to be very valuable in computing decision making policies for problems where structure exists, but is hard to define manually, which is often the case in practice. Indeed, many recent RL success stories, such as mastering the game of Go, managing resources in the complex game of StarCraft, and state-of-the-art continuous control of racing drones, have all used the simple MDP model combined with powerful deep learning methods~\cite{silver2016mastering,vinyals2019grandmaster,kaufmann2023champion}.

There are two other strong modelling assumptions in the MDP:  (1) all uncertainty in decision making is limited to the randomness of the (Markov) state transitions, and (2) the objective can only be specified using rewards. We next discuss these two design choices in more detail.

%randomness

\section{Reasoning Under Uncertainty}

A main objective of planning and learning is to facilitate reasoning under uncertainty. Uncertainty can come in many forms, as humans often encounter every day: when playing a board game, we do not know what our opponent will do; when rolling a dice, we do not know what will be the outcome; when folding laundry, it is likely that an accurate physical model of the cloth is not available; when driving in a city, we may not observe cars hidden behind a corner; etc. 


% \AT{I think we should work on connecting this section better to the preceding text. Right now it's not clear to me. Is it supposed to answer: "Why do we assume that we know the model?" and the answer is either: "we don't, and this uncertainty is part of the stochasticity of the MDP" or "we don't, and in the learning chapter we'll show you what you can do to learn the model". If this is what you meant, then we can clarify it this way.}


Here we list common types of uncertainty, and afterwards discuss how they relate to planning with MDPs. 

    {\em Aleatoric\footnote{The term comes from the Latin word ``alea'', which means dice or a game of chance.} uncertainty}: Handling inherent randomness that is part of the model. For example, a model can contain an event which happens with a given probability. For example, in the board game backgammon, the probability of a move is given by throwing two dices.
    
    {\em Epistemic\footnote{The term is derived from the Greek word ``episteme'', meaning knowledge.} uncertainty}: Dealing with lack of knowledge about the model parameters. Sometimes we do not know what are the exact model parameters because more interaction is needed to learn them (this is addressed in the learning part of this book). Sometimes we have a nominal model and the true parameters are only  revealed at runtime (this is addressed within the robust MDP framework; see \cite{nilim2005robust}). Sometimes our model is too coarse or simply incorrect -- this is known as {\em model misspecification}.
    
    % {\em Model misspecification}: Addressing incorrect assumptions. In many cases, we do not know what is the true model and not just the parameters. In these cases, we have to deal with model misspecification. 
    
    {\em Partial observability}: Reasoning with incomplete information concerning the true state of the system. There are many problems where  we just do not have an accurate measurement that can help us predict the future and instead we get to observe partial information concerning the true state of the system. Some would argue that all real problems have some elements of partial observability in them.

% \AT{Where does adversarial opponent fit here?}

We emphasize that for planning and for learning, a model could combine all types of uncertainty. The choice of which type of uncertainty is an important  design choice.  

The MDP model that we focus on in the planning chapter only accommodates aleatoric uncertainty, through the stochastic state transitions. While this may appear to be a strong limitation, MDPs have proven useful for dealing with more general forms of uncertainty. For example, in the learning chapters, we will ask how to update an MDP model from interaction with the environment, to potentially reduce epistemic uncertainty. For board games, even though MDPs cannot model an adversary, assuming that the opponent is stochastic helps find a robust policy against various opponent strategies. Moreover, by using the concept of \textit{self play} -- an agent that learns to play against itself and continually improve -- RL has produced the most advanced AI agents for several games, including Chess and Go. For partially observable systems, a fundamental result shows that taking the full observation history as the `state', results in an MDP model for the problem (albeit with a huge state space). 




\section{Objective Optimization}

A central assumption in planning is that an immediate reward function is given to the decision maker and the objective is to maximize some sort of expected cumulative discounted (or total or average) reward. 
From the perspective of the planner, this makes the planner's life easy as the objective is defined in a formal manner.
Nevertheless, in many applications, much of the problem is to engineer a ``right'' reward function. This may be done by understanding the specifications of the problem, or from data of desired behavior, a problem known as \textit{Inverse Reinforcement Learning}~\cite{ng2000algorithms}. 

Specifically, the mere existence of a reward function implies that every aspect of the decision problem can be converted into a single currency. For example, in a communication network minimizing power and maximizing bit rate may be hard to combine into a single reward function. 
Moreover, even when  all aspects in expectation of a problem can be amortized with a single reward function the decision maker may have other risk aspects in mind, such as resilience to rare events. 
We emphasize that the reward function is a design choice made by the decision maker.

In some cases, the reward stream is very sparse. For example, in board game the reward is often obtained only at the end of the game in the form of a victory or a loss. While this does not pose a conceptual problem, it may lead to practical problems as we will discuss later in the book. A conceptual solution here is to use ``proxy rewards".

A limitation of the Markov decision process planning model is
the underlying assumption that preferences can be succinctly represented through reward functions. While in principle, any preference among trajectories can be 
represented using a reward function, by extending the state space to include all history, this may be cumbersome and may require a much larger state space. Specifically, the discount factor which is often assumed a part of the problem specifications, represent a preference between short-term and long-term objectives. Such preferences are often arbitrary.

We finally comment that the assumption that there exists a scalar reward we optimize (through a long term objective) does not hold in many problems. Often, we have several potentially contradicting objectives. For example, we may want to minimize power consumption while maximizing throughput in communication networks. In general part of the reward function engineering pertains to balancing different objectives, even if they are not measured in the same way (``adding apples and oranges''). A different approach is to embrace the multi-objective nature of the decision problems through constrained Markov decision processes \cite{altman2021constrained}, or using other approaches~\cite[e.g.,][]{mannor2004geometric}. 

Nevertheless, MDPs with their single reward function have proven useful in many practical domains, as the availability of strong algorithms for solving MDPs effectively allow the system engineer to tweak the reward function manually to fit some hard-to-quantify desired behavior.

% \begin{itemize}
%     \item Reward design: Epistemological challenges in defining "right" objectives
%     \item Proxy rewards: Implications of optimizing indirect measures
%     \item Long-term vs. short-term objectives: Balancing immediate and future knowledge, the dubious role of the discount factor
%     \item Multi-objective reinforcement learning: Handling competing  goals
%     \item Reality gap: Transferring knowledge from simulations to real-world
%     \item Reward specification: Challenges in defining and pursuing goals
%     \item Inverse reinforcement learning: Inferring objectives from observed behavior
% \end{itemize}


\section{Importance of Small (Finite) Models}

The next few chapters, and indeed much of the literature, explicitly assume that the models are finite (in terms of actions and states) and even practically small. While this is certainly justified from a pedagogical perspective there are additional reasons for that make small models relevant. 


Small models are more interpretable than large ones: it is often the case that different state capture particular meanings and hence lead to more explainable policies. For example, in inventory control problems, the dynamic programming techniques that we will study can show that for certain simplified problem instances, an \textit{optimal} strategy has the structure of a threshold policy -- if the inventory is below some certain threshold then replenish, otherwise do not. Such observations about the structure of optimal policies often inform the design of policies for more complex scenarios.

The language and some fundamental concepts we shall develop for small models, such as the value function, value iteration and policy iteration algorithms, and convergence of stochastic approximation, will also carry over to the learning chapters, which deal with large state spaces and approximations.

% Balancing model complexity with explanatory power may be considered through the lenses of Ocaam's razor: the simplest model that can explain a phenomenon, in this case the control policy.
% Moreover, when learning is involved, the number of model parameters is smaller, so learning may be more efficient. 

% Small models have tremendous computational advantages: storing their parameters, computing optimal policies offline and in runtime are all substantially more tractable when the models are small.

% \YM{Very important!}
% \begin{itemize}
%     \item Occam's Razor in RL: Balancing model complexity with explanatory power
%     \item Finite MDPs: Epistemological implications of working with limited state spaces
%     \item Sample efficiency: Learning effectively from limited experiences
%     \item Interpretability of small models: Enhanced understanding and explainability
%     \item Generalization in finite models: Extracting broad knowledge from limited representations
%     \item Computational benefits: Tractability and scalability considerations
%     \item Latent models
% \end{itemize}


% \section{Representation of Knowledge}
% \YM{Not sure where state representation goes. Note that up to here they have only tabular setting, so no representation.}
% \SM{Leave to the last soft chapter}

% \begin{itemize}
%     \item State representation: Abstracting relevant features from environment
%     \item Policy-based knowledge: Mapping states to actions
%     \item Value-based knowledge: Estimating long-term rewards
%     \item Model-based knowledge: Explicit representation of environment dynamics
% \end{itemize}




% \section{Robustness and Adaptability}
% \YM{Again, it is out of scope for the book.} 
% \SM{Leave to the last soft chapter}

% \begin{itemize}
%     \item Distributional shift: Handling differences between training and deployment
%     \item Adversarial robustness: Maintaining performance under perturbations
%     \item Continual learning: Adapting to changing environments over time
% \end{itemize}

