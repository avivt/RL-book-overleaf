

\section{Stochastic Bandits and Regret Minimization}

We consider a simplified model of an MDP where there is only a
single state and a fixed set $A$ of $k$ actions (a.k.a., arms). We
consider a finite horizon problem, where the horizon is $T$.
Clearly, the planning problem is trivial, simply select the action
with the highest expected reward. We will concentrate on the
learning perspective, where the expected reward of each action are
unknown.

At each round $1\leq t\leq T$ the player selects and executes an
action. After executing the action, the player observes the reward
of the action. However, the rewards of the other actions in $A$ are
not revealed to the player.

The reward for action $i$ at round $t$ is denoted by $r_{t}(i)\sim
D_{i}$, where the support of the reward distribution $D_{i}$ is
$[0,1]$. We assume that the rewards are i.i.d. (independent and
identically distributed).

\paragraph{Motivation}
\begin{enumerate}
\item \textbf{News}: a user visits a news site and is presented with a news
header. The user either clicks on this header or not. The goal of
the website is to maximize the number of clicks. So each possible
header is an action in a bandit problem, and the clicks are the
rewards
\item \textbf{Medical Trials}: Each patient in the trial is prescribed one
treatment out of several possible treatments. Each treatment is an
action, and the reward for each patient is the effectiveness of the
prescribed treatment.
\item \textbf{Ad selection}: In website advertising, a user visits a webpage,
and a learning algorithm selects one of many possible ads to
display. If an advertisement is displayed, the website observes
whether the user clicks on the ad, in which case the advertiser pays
some amount $v_{a} \in[0, 1]$. So each advertisement is an action,
and the paid amount is the reward.
\end{enumerate}

\paragraph{Model}
\begin{itemize}
\item A set of actions $A=\left\{ a_{1}\dots,a_{k}\right\} $
\item Each action $a_{i}$ has a reward distribution $D_{i}$ over
$[0,1]$.
\item The expectation of distribution $D_{i}$ is:
\[
\mu_{i}=E_{X\sim D_{i}}\left[X\right]
\]
\item $\mu^{*}= \max_{i}\mu_{i}$ and $a^*=\arg\max_i\mu_i$.
\item $a_{t}$ is the action the learner chose at round $t$
\[
Regret=\max_{i\in A}{\displaystyle
\sum_{t=1}^{T}\underbrace{r_{t}(i)}_{\text{Random
variable}}}-{\displaystyle
\sum_{t=1}^{T}\underbrace{r_{t}(a_{t})}_{\text{Random variable}}}
\]
\[
\begin{array}{ccc}
\text{Pseudo Regret} & = & {\displaystyle \max_{i}}E\left[{\displaystyle \sum_{t=1}^{T}r_{t}(i)}\right]-E\left[{\displaystyle \sum_{t=1}^{T}r_{t}(a_{t})}\right]\\
\\
 & = & \mu^{*}\cdot T-{\displaystyle \sum_{t=1}^{T}\mu_{a_{t}}}
\end{array}
\]
\end{itemize}

\ymignore{
\begin{leftbar}
\section{Sub-Gaussian Random Variable}

A random variable $X$ is called $\sigma^{2}-sub-gaussian$ if for any
$\lambda\in\mathbb{R}$:

\[
E\left[e^{\lambda X}\right]\le e^{\sigma^{2}\lambda^{2}/2}
\]

\paragraph{Examples}
\begin{enumerate}
\item $X\sim N\left(0,\sigma^{2}\right)$, $E\left[e^{\lambda X}\right]=e^{\sigma^{2}\lambda^{2}/2}$
$\Rightarrow$ X is $\sigma^{2}-sub-gaussian$
\item X s.t $\begin{cases}
E\left[X\right]=0\\
\left|X\right|\le B
\end{cases}$, then X is $B^{2}-sub-gaussian$
\end{enumerate}

\subsection{Properties of $\sigma^{2}-sub-gaussian$ random variable $X$}
\begin{itemize}
\item $E\left[X\right]=0$, $Var\left(X\right)\le\sigma^{2}$
\item $c\cdot X$ is $c^{2}\sigma^{2}-sub-gaussian$
\item If $X_{1}\dots,X_{m}$ are $\sigma^{2}-sub-gaussian$ then $S={\displaystyle \sum_{i=1}^{m}X_{i}}$
is $m\sigma^{2}-sub-gaussian$\\
 $\frac{1}{m}S=\frac{1}{m}{\displaystyle \sum_{i=1}^{m}X_{i}}$ is
$\frac{\sigma^{2}}{m}-sub-gaussian$
\end{itemize}

\begin{theorem}
\label{thm:sub-gaus}
Let X be $\sigma^{2}-sub-gaussian$ random
variable, then
$Pr[X\geqslant\epsilon]\le\exp(-\frac{\epsilon^{2}}{2\sigma^{2}})$.
\end{theorem}

\begin{proof}
\[
\begin{array}{ccc}
Pr[X\geqslant\epsilon] & = & Pr[e^{\lambda X}\geqslant e^{\lambda\epsilon}]\\
\\
 & \le & \frac{E[e^{\lambda X}]}{e^{\lambda\epsilon}}\\
\\
 & \le & \exp(\sigma^{2}\lambda^{2}/2-\lambda\epsilon)
\end{array}
\]
where we used Markov's inequality for the first stage and the fact
that the random variable is $\sigma^{2}-sub-gaussian$ for the
second.

If we choose $\lambda=\frac{\epsilon^{2}}{\lambda^{2}}$, then we
get:
\[
\begin{array}{ccc}
Pr[X\geqslant\epsilon] & \le & \exp(-\frac{\epsilon^{2}}{2\sigma^{2}})\end{array}
\]
\end{proof}
\end{leftbar}

\subsection{Hoeffding's inequality}
}

We will use extensively the following concentration bound.

\begin{theorem}[Hoeffding's inequality]
\label{thm:hoeffding}
%
Given $X_{1},\dots,X_{m}$ i.i.d random
variables s.t $X_{i}\in[0,1]$ and $E[X_{i}]=\mu$.
\[
\begin{array}{ccc}
Pr[\underbrace{\frac{1}{m}{\displaystyle \sum_{i=1}^{m}X_{i}-\mu}}_{\frac{1}{m}S}\ge\epsilon] & \le & \exp(-\frac{\epsilon^{2}m}{2})\end{array}
\]
\end{theorem}

\ymignore{
\begin{leftbar}
This is a direct conclusion from Theorem~\ref{thm:sub-gaus}. Let
$\bar{X_{i}}=X_{i}-\mu$. Using the facts that $E[\bar{X_{i}}]=0$
and:
\[
\begin{array}{ccc}
X_{i}\in[0,1] & \Rightarrow & |\bar{X_{i}|}\le1\\
\\
 & \Rightarrow & X_{i}\text{ is 1-sub-gaussian}
\end{array}
\]
\end{leftbar}
}

\subsection{Warmup: Full information $k=2$ }

We start with a simple case where there are two actions and we
observe the reward of both actions at each time $t$. We will analyze
the greedy policy, which selects the action with the higher average
reward (so far).

The greedy policy at time $t$ does the following:
\begin{itemize}
\item We observe $\big\langle r_{t}(1),r_{t}(2)\big\rangle$
\item Define
\[
avg_{t}(i)=\frac{1}{t} \sum_{\tau=1}^{t}r_{\tau}(i)
\]
\item In time $t+1$ we choose:
\[
a_{t+1}=\arg\max_{i\in\{1,2\}}avg_{t}(i)
\]
\end{itemize}


We now would like to compute the expected regret of the greedy
policy. W.l.o.g., we assume that $\mu_{1}\ge\mu_{2}$, and define
$\Delta=\mu_{1}-\mu_{2}\ge0$.
\[
\text{Pseudo Regret}= \sum_{t=1}^{\infty}(\mu_{1}-\mu_{2})
\Pr\left[avg_{t}(2)\ge avg_{t}(1)\right]
\]
Clearly, at any time $t$,
\[
E[avg_{t}(2)-avg_{t}(1)]=\mu_{2}-\mu_{1}=-\Delta
\]
We can define a random variable $X_t=r_t(2)-r_t(1)+\Delta$ and
$E[X_t]=0$. Since $(1/t)\sum_t X_t = avg_t(2)- avg_t(1)+\Delta$, by
Theorem~\ref{thm:hoeffding}
\[
Pr[S_2(t)\geq S_1(t)]= Pr\left[avg_{t}(2)-avg_{t}(1)+\Delta
\ge\Delta\right]\le e^{-\Delta^{2}\frac{t}{2}}
\]
We can now bound the regret as follows,

\begin{align*}
E\left[\text{Pseudo Regret}\right] & = &
 \sum_{t=1}^{\infty}\Delta\Pr\left[S_{2}(t)\ge S_{1}(t)\right]\\
 & \le &  \sum_{t=1}^{\infty}\Delta e^{-\Delta^{2}\frac{t}{2}}\\
 & \le &  \int_{0}^{\infty}\Delta e^{-\Delta^{2}\frac{t}{2}}dt\\
 & = & \left[\frac{2}{\Delta}e^{-\Delta^{2}\frac{t}{2}}\right]_{0}^{\infty}\\
 & = & \frac{2}{\Delta}
\end{align*}
Notice that this regret bound does not depend on $T$!

\subsection{Stochastic Multi-Arm Bandits}

%\subsection{Bandits}

We will now see that we cannot get a regret that does not depend on
$T$ for the bandits case. Considering the following example:
\[
a_{1}\sim Br\left(\frac{1}{2}\right)\]

For action $a_2$ there are two alternatives, each with probability
$1/2$,
\[
a_{2}\sim Br\left(\frac{1}{4}\right) \left(w.p. \frac{1}{2}\right)
\qquad or \qquad a_{2}\sim Br\left(\frac{3}{4}\right) \left(w.p.
\frac{1}{2}\right)
\]

Assume by way of contradiction
\[
E\left[ \sum_{i \in \{1,2\}} \Delta_i |T_i|\right] = E\left[Pseudo
Regret\right]=R
\]

where $R$ does not depend on $T$. \\ By Markov inequality:
\[
Pr\left[Pseudo Regret\ge2R\right]\le\frac{1}{2}
\]

Since $\mu_1$ is known, an optimal algorithm will first check $a_2$
in order to decide which action is better and stick with it.

Assuming $\mu_2 = \frac{1}{4}$, and the algorithm decided to stop
playing $a_2$ after $M$ rounds, Then:
\[
Pseudo Regret = \frac{1}{4}M
\]
Thus,
\[
Pr\left[Pseudo Regret\ge 2R \right] = Pr\left[ M\ge 8R
\right]\le\frac{1}{2}
\]
And,
\[
Pr\left[M < 8R \right]>\frac{1}{2}
\]
Hence, the probability that after $8R$ rounds, the algorithm will
stop playing $a_2$ (if $\mu_2 = \frac{1}{4}$) is at least
$\frac{1}{2}$. This implies that there is some sequence of $8R$
outcomes which will result is stopping to try action $a_2$. For
simplicity, assume that the sequence is the all zero sequence.

Assume $\mu_2 = \frac{3}{4}$, but all $8R$ first rounds, playing
$a_2$ yield the value zero (which happens with probability
$\left(\frac{1}{4}\right)^{8R}$). We assumed that after $8R$ zeros
for action $a_2$ the algorithm will stop playing $a_2$, even though
it is the preferred action. In this case, we will get:
\[
Pseudo Regret = \frac{1}{4} (T - M) \approx \frac{1}{4}T
\]

The expected Pseudo Regret is,
\[
E\left[Pseudo Regret\right] = R \geq
\underbrace{\frac{1}{2}}_{a_{2}\sim Pr(Br\left(\frac{3}{4}\right))}
\cdot \underbrace{\left(\frac{1}{4}\right)^{8R}}_{Pr(all \; 0 |
a_{2}\sim Pr(Br\left(\frac{3}{4}\right))} \cdot (T - 8R) \approx
e^{-O(R)}T
\]

Which implies that:
\[
R=\Omega\left(\log T\right)
\]
Contrary to the assumption that $R$ does not depend on $T$.
%\footnote{More formally, after $8R$ steps, there exists some
%sequence of outcomes that cause the algorithm to switch to action
%$a_2$. The probability of that sequence is at least
%$\left(\frac{1}{4}\right)^{8R}$.}

\topic{Explore-Then-Exploit}{random}

\begin{enumerate}
%
\item We choose a parameter $M$.
% time frame $kM$ to explore.
%
For $M$ phases we choose each action once (for a total of $kM$
rounds of exploration).
%
\item After $kM$rounds we always choose the action that had highest
average reward during the explore phase.
\end{enumerate}

Define:
\begin{align*}
T_{j}&= \left\{ t:a_t=j,t\le k\cdot M\right\}\\
\hat{\mu}_{j}&=\frac{1}{M} \sum_{t\in T_{j}}r_{j}(t)\\
\mu_{j}&=E[r_{j}(t)]\\
\Delta_{j}&=\mu^{*}-\mu_{j}
\end{align*}
where $\Delta_j$ is the difference in expected reward of action $j$
and the optimal action.

We can now write the regret as a function of those parameters:
\[
E\left[\text{Pseudo regret}\right]=\underbrace{{\displaystyle
\sum_{j=1}^{k}\Delta_{j}\cdot
M}}_{Explore}+\underbrace{\left(T-k\cdot M\right){\displaystyle
\sum_{j=1}^{k}\Delta_{j}Pr\left[j=\arg\max_{i}\hat{\mu}_{i}\right]}}_{Exploit}
\]

For the analysis define:
\[
\lambda=\sqrt{\frac{8\log T}{M}}
\]

By Theorem~\ref{thm:hoeffding} we have
\[
\Pr\left[\left|\hat{\mu}_{j}-\mu_{j}\right|\ge\lambda\right]  \le
2e^{-\frac{\lambda^2 M}{2}}=\frac{2}{T^{4}}
\]
which implies (using the union bound) that
\[
\Pr\underbrace{\left[\exists_{j}:\left|\hat{\mu}_{j}-\mu_{j}\right|\ge\lambda\right]}_{B}
 \le  \frac{2k}{T^{4}}\underset{for\,k\leq T}{\leq}\frac{2}{T^{3}}
\]

Define the ``bad event''
$B=\{\exists_{j}:\left|\hat{\mu}_{j}-\mu_{j}\right|\ge\lambda\}$. If
$B$ did not happen then for each action $j$, such that
$\hat{\mu}_{j}\ge\hat{\mu}^{*}$, we have
\[
\mu_{j}+\lambda\ge\hat{\mu}_{j}\ge\hat{\mu}^{*}\ge\mu^{*}-\lambda
\]
therefore:
\[
2\lambda\ge\mu^{*}-\mu_{j}=\Delta_{j}
\]
and therefore:
\[
\Delta_{j}\le2\lambda
\]

Then, we can bound the expected regret as follows:
\begin{align*}
E[PseudoRegret] & \leq  \underbrace{\left({\displaystyle
\sum_{j=1}^{k}\Delta_{j}}\right)M}_{Explore}+\underbrace{\left(T-k\cdot
M\right)\cdot2\lambda}_{\text{B didn't
happen}}+\underbrace{\frac{2}{T^{3}}\cdot T}_{\text{B happened}}\\
 & \leq  k\cdot M+2\cdot\sqrt{\frac{8\log T}{M}}\cdot T+\frac{2}{T^{2}}
\end{align*}

If we optimize the number of exploration phases $M$ and choose $M=T^{\frac{2}{3}}$,
we get:
\[
k\cdot T^{\frac{2}{3}}+2\cdot\sqrt{8\log T}\cdot
T^{\frac{2}{3}}+\frac{2}{T^{2}}
\]
which is sub-linear but more than the $O(\sqrt{T})$ rate we would
expect.

\section{Improved Regret Minimization Algorithms}

We will look at some more advanced algorithms that mix the
exploration and exploitation.

Define:

$n_{t}(i)$ - the number of times we chose action $i$ by round $t$

$\hat{\mu}_{t}(i)$ - the average reward of action $i$ so far, that
is:
\[
\hat{\mu}_{t}(i)={\displaystyle
\sum_{t=1}^{T}r_{i}(t)}\mathbb{I}\left(a_{t}=i\right)\frac{1}{n_{i}(t)}
\]
Notice that $n_{i}(t)$ is a random variable and not a number!

We would like to get the following result:
\[
\Pr\left[\left|\hat{\mu}_{t}(i)-\mu_{i}\right|\le\underbrace{\sqrt{\frac{8\log
T}{n_{i}(t)}}}_{\lambda_{t}(i)}\right]\ge1-\frac{2}{T^{4}}
\]

We would like to look at the $m^{th}$ time we sampled action $i$:
\[
\hat{\mathbb{V}}_{m}(i)=\frac{1}{m} \sum_{\tau=1}^{m}r_{i}(t_{\tau})
\]
Where the $t_{\tau}$'s are the rounds when we chose action $i$

Now we fix $m$ and get:
\[
\forall{i}\forall{m}\;\;\;
\Pr\left[\left|\hat{\mathbb{V}}_{m}(i)-\mu_{i}\right|\le\sqrt{\frac{8\log
T}{m}}\right]\ge1-\frac{2}{T^{4}}
\]
and notice that $\hat{\mu}_{t}(i)\equiv\hat{\mathbb{V}}_{i}(m)$ when
$m=n_{i}(t)$.

Define the ``good event'' $G$:
\[
G=\left\{ \forall_{i}\forall_{t}\left|\hat{\mu}_{i}(t)-\mu_{i}\right|\le\lambda_{i}(t)\right\}
\]
The probability of $G$ is,
\[
Pr\left(G\right)\ge1-\frac{2}{T^{2}}
\]

\section{Refine Confidence Bound}

Define the upper confidence bound:
\[
UCB_{t}(i)=\hat{\mu}_{t}(i)+\lambda_{t}(i)
\]
and similarly, the lower confidence bound:
\[
LCB_{t}(i)=\hat{\mu}_{t}(i)-\lambda_{t}(i)
\]
if $G$ happened then:
\[
\forall{i}\forall{t}\;\;\;\mu_{i}\in\left[LCB_{t}(i),UCB_{t}(i)\right]
\]
Therefore:
\[
Pr\biggl[\forall{i}\forall{t}\;\;\;
\mu_{i}\in\left[LCB_{t}(i),UCB_{t}(i)\right]\biggr]\ge1-\frac{2}{T^{2}}
\]

\subsection{Successive Elimination}

We maintain a set of actions S.

Initially $S=A$

In each phase:
\begin{itemize}
\item We try every $i\in S$ once
\item For each $j\in S$ if there exists $i\in S$ such that:
\[
UCB_{t}(j)<LCB_{t}(i)
\]
 We remove $j$ from $S,$ that is we update:
\[
S\leftarrow S-\left\{ j\right\}
\]
\end{itemize}
We will get the following results:
\begin{itemize}
\item As long as action $i$ is still in $S$, we have tried action $i$ exactly the
same number of times as all of any other action $j\in S$.
\item The best action, under the assumption that the event $G$ holds, is never eliminated
from $S$.
\end{itemize}
Under the assumption of $G$ we get:

\[
\mu^{*}-2\lambda\le\hat{\mu}^{*}-\lambda=LCB_{*}<UCB_{i}=\hat{\mu_{i}}+\lambda\leq\mu_{i}+2\lambda
\]

Where $\lambda=\lambda_{i}=\lambda^{*}$ because we have chosen
action $i$ and the best action the same number of times so far.

Therefore, assuming event $G$ holds,
\[
\Delta_{i}=\mu^{*}-\mu_{i}\le4\lambda=4\sqrt{\frac{8\log
T}{n_{t}(i)}}
\]
\[
\Rightarrow\;\;\; n_{T}(i)\le\frac{c}{\Delta_{i}^{2}}\log T
\]

This implies that
\begin{align*}
E\left[\text{Pseudo Regret}\right]  = &  \sum_{i=1}^{k}\Delta_{i}n_{i}(t)\\
  \le &  \sum_{i=1}^{k}\frac{c}{\Delta_{i}}\log T
  +\underbrace{\frac{2}{T^{2}}\cdot T}_{\text{The bad event}}
\end{align*}

meaning that the expected pseudo regret is bounded by $O\left(\frac{1}{T}\right)$.

\subsection{Upper confidence bound (UCB)}

The UCB algorithm simply uses the UCB bound. The algorithm works as
follows:
\begin{itemize}
\item We try each action once (for a total of $k$ rounds)
\item Afterwards we choose:
\end{itemize}
\[
a_{t}=\arg\max_{i}UCB_t(i)
\]

If we chose action $i$ then, assuming $G$ holds, we have
\[
UCB_{t}(i)  \ge  UCB_t(a^*)\geq\mu^{*}
\]
where $a^*$ is the optimal action.

Using the definition of UCB and the assumption that $G$ holds, we
have
\[
UCB_t(i)=\hat{\mu}_{t}(i)+\lambda_{t}(i)\le\mu_{i}+2\lambda_{t}(i)
\]
Since we selected action $i$ at time $t$ we have
\[
\mu_{i}+2\lambda_{t}(i)\ge\mu^{*}
\]
Rearranging, we have,
\[
2\lambda_t(i)\ge\mu^{*}-\mu_{i}=\Delta_{i}
\]
Each time we chosen action $i$, we could not have made a very big
mistake because:
\[
\Delta_{i}\leq\text{ }2\cdot\sqrt{\frac{8\log T}{n_{t}(i)}}
\]

And therefore if $i$ is very far off from the optimal action we
would not choose it too many times. We can bound the number of time
action $i$ is used by,
\[
n_{t}(i)\leq\frac{c}{\Delta_{i}^{2}}\log T
\]

And over all we get:
\begin{align*}
E\left[\text{Pseudo Regret}\right] & =
\sum_{i=1}^{k}\Delta_{i}E\left[n_{t}(i)\right]+\underbrace{\frac{2}{T^{2}}\cdot
T}_{\text{The bad event}}
\\
 & \le  \sum_{i=1}^{k}\frac{c}{\Delta_{i}}\cdot\log T+\frac{2}{T}
\end{align*}

\section{Best Arm Identification}

We would like to identify the best action, or an almost best action.
We can define the goal in one of two ways.

\paragraph{PAC criteria }
An action $i$ is $\epsilon$-optimal if  $\mu_i\geq \mu^*-\epsilon$. The PAC criteria is
that, given $\epsilon,\delta>0,$, with probability at least
$1-\delta$, find an $\epsilon$ optimal action.

\paragraph{Exact identification}
Given $\Delta\le\min_{i\neq a^*}\mu^{*}-\mu_{i}$ (for every suboptimal action $i$),
find the optimal action $a_{*}$, with probability at least
$1-\delta$.

\subsection{Naive Algorithm (PAC criteria):}

We sample each action $i$ for
$m=\frac{8}{\epsilon^{2}}\log\frac{2k}{\delta}$ times, and return
$a= \arg\max_{i}\hat{\mu}_{i}$ .

For rewards in $[ 0 ,1]$, then, by Theorem \ref{thm:hoeffding}, for
every action $i$ we have
\[
Pr\left[\underbrace{\left|\hat{\mu}_{i}-\mu_{i}\right|>\frac{\epsilon}{2}}_{\text{bad
event}}\right]\le 2
e^{-\left(\frac{\epsilon}{2}\right)^{2}m/2}=\frac{\delta}{k}
\]
By union bound we get:
\[
Pr\left[\exists_{i}\left|\hat{\mu}_{i}-\mu_{i}\right|>\frac{\epsilon}{2}\right]\le\delta
\]

If the bad event
$B=\{\exists_{i}\left|\hat{\mu}_{i}-\mu_{i}\right|>\frac{\epsilon}{2}\}$
did not happen, then both: (1)
$\mu^{*}-\frac{\epsilon}{2}\le\hat{\mu}^{*}$ and (2)
$\mu_{i}+\frac{\epsilon}{2}\le\hat{\mu}_{i}$.

This implies,
\[
\Rightarrow\text{ }\mu_{i}+\frac{\epsilon}{2}\ge\hat{\mu}_{i}\ge\hat{\mu}^{*}\ge\mu^{*}-\frac{\epsilon}{2}
\]
\[
\Rightarrow\text{ }\epsilon\ge\mu^{*}-\mu_{i}
\]

And therefore $a={\displaystyle \arg\max_{i}\hat{\mu}_{i}}$ is the
optimal action in probability $1-\delta$.

We would like to slightly improve the sample size of this algorithm

\subsection{Median Algorithm}

The idea: the algorithm runs for $l$ phases, after each phase we
eliminate half of the actions. This elimination allows us to sample
each action more times in the next phase which makes eliminating the
optimal action less likely.

\begin{algorithm}
\textbf{Input}: $\epsilon,\delta>0$

\textbf{Output}: $\bar{a}\in A$

\textbf{Init}: $S_{1}=A$, $\epsilon_{1}=\frac{\epsilon}{4}$,
$\delta_{1}=\frac{\delta}{2}$, $l=1$

\textbf{Repeat}:

~~~~~~$\forall_{i}\in S_{l}$, sample action i, $\frac{1}{\left(\frac{\epsilon_{l}}{2}\right)^{2}}\log\left(\frac{3}{\delta_{l}}\right)$
times

~~~~~~$\hat{\mu}_{i}\leftarrow\text{mean (only of samples during the \ensuremath{l^{th}}phase)}$

~~~~~~$\text{median}_{l}\leftarrow median\left\{ \hat{\mu}_{i}:i\in S_{l}\right\} $

~~~~~~$S_{l+1}\leftarrow\left\{ i\in S_{l}:\hat{\mu}_{i}\ge\text{median}_{l}\right\} $

~~~~~~$\epsilon_{l+1}\leftarrow\frac{3}{4}\epsilon_{l}$

~~~~~~$\delta_{l+1}\leftarrow\frac{\delta_{l}}{2}$

~~~~~~$l\leftarrow l+1$

\textbf{Until} $\left|S_{l}\right|=1$

\caption{Best Arm Identification}
\end{algorithm}

\paragraph{Complexity:}

During phase $l$ we have $\left|S_{l}\right|=\frac{k}{2^{l-1}}$
actions.
\[
\epsilon_{l}=\frac{3}{4}\epsilon_{l-1}=\frac{\epsilon}{4}\left(\frac{3}{4}\right)^{l-1},\text{
}\delta_{l}=\frac{\delta}{2^{l}}
\]
\[
\Rightarrow\text{ }{\displaystyle \sum\epsilon_{l}\le\epsilon},\text{ }{\displaystyle \sum\delta_{l}\le\delta}
\]

The total number of samples is therefore:

\begin{align*}
 \sum_l |S_l| \cdot \frac{4}{\epsilon_l^2}\log \frac{3}{\delta_l} &=
\sum_{l}\frac{k}{2^{l-1}}\frac{64}{\epsilon^{2}}\left(\frac{16}{9}\right)^{l-1}\log\frac{3\cdot2^{l}}{\delta}\\
& =  {\displaystyle \sum_{l}k\left(\frac{8}{9}\right)^{l-1}\left[c\cdot\frac{\log\frac{1}{\delta}}{\epsilon^{2}}+\frac{\log3}{\epsilon^{2}}+\frac{l}{\epsilon^{2}}\right]}\\
\\
 & =  O\left(\frac{k}{\epsilon^{2}}\log\frac{1}{\delta}\right)
\end{align*}


\paragraph{Correctness:}

\begin{theorem}
$Pr\left[\underbrace{{\displaystyle \max_{j\in
S_{l}}\mu_{j}}}_{\text{action
\ensuremath{l}}}\le\underbrace{{\displaystyle \max_{j\in
S_{l+1}}\mu_{j}}}_{\text{action
\ensuremath{l+1}}}+\epsilon_{l}\right]\ge1-\delta_{l}$
\end{theorem}

\begin{proof}
%We do the proof for $l=1$, general $l$ is similar. 
Let $\mu^*_l=\max_{j\in S_l} \mu_j$.
Define $E_{l}=\left\{\hat{\mu}^{*}_l<\mu^{*}_l-\frac{\epsilon_{1}}{2}\right\}$.
We have  $Pr\left[E_{l}\right]\le\frac{\delta_{l}}{3}$. If $E_{l}$ did not happen, we define a bad set:
\[
\text{Bad}=\left\{ j\text{ }:\text{ }\mu^{*}-\mu_{j}\ge\epsilon_{l},\text{ }\hat{\mu}_{j}\ge\hat{\mu}^{*}\right\}
\]
Consider an action $j$ such that $\mu^*-\mu_j\geq \epsilon_l$, then:
\[
Pr[\hat{\mu}_{j}\geq
\hat{\mu}^{*}|\underbrace{\hat{\mu}^{*}\ge\mu^{*}-\frac{\epsilon_{1}}{2}}_{\urcorner
E_{l}}\big] \le
Pr[\hat{\mu}_{j}\ge\mu_{j}+\frac{\epsilon_{1}}{2}|\urcorner E_{l}]
\le  \frac{\delta_{1}}{3}
\]
Note that the probability is not negligible. We will show that it cannot happen to too many such actions. We will bound the expectation of the size of $Bad$,
\[
E[|\text{Bad}||\urcorner E_{l}]\le k\frac{\delta_{1}}{3}
\]
 with Markov's inequality we get:
\[
Pr\left[\left|\text{Bad}\right|\ge\frac{k}{2}\right|\urcorner E_{l}]
 \le  \frac{E\left|\text{Bad}\right|}{k/2}
  =  \frac{2}{3}\delta_{l}
\]
 with probability $1-\delta_{l}$: $\hat{\mu}^{*}\ge\mu^{*}-\frac{\epsilon_{1}}{2}$ and $\left|\text{Bad}\right|\le\frac{k}{2}$. Therefore: $\exists_{j}\notin\text{Bad}$ and $j\in S_{l+1}$.
\end{proof}
