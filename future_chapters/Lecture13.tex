
Today lecture will have two different topics. The first is a
different model for large state pace MDP, called {\em generative
model}. The second is {\em Inverse Reinforcement Learning} (IRL) and
{\em Apprenticeship Learning}. In both we are given the optimal
policy but have no access to the rewards. In IRL we try to build the
reward function which is consistent with it. In AL we try to build a
policy which has a similar return for any reward function.

\section{Generative Model}

A generative model is a different way to model a large state space
MDP. Rather then assuming a succinct representation, we are given a
``black box'', called generator, that can simulate the model. The
generator gets an inputs a state $s\in S$ and an action $a\in A$ and
outputs a next state $s'\in S$ distributed according to $p(\cdot
|s,a)$ and a reward $r$ distributed according to $R(s,a)$. We can
also extend the model to describe a POMDP. For a POMDP, the inputs
are a state $s\in S$ and an action $a\in A$ and outputs a next state
$s'\in S$ distributed according to $p(\cdot |s,a)$, a reward $r$
distributed according to $R(s,a)$, and an observation $o$
distributed according to $O(\cdot|s',a)$.

\subsection{Policy Evaluation}

Recall that for policy evaluation, we are given a policy $\pi$ and
would like to estimate its return from the start state, i.e.,
$V^\pi(s_0)$.

We will consider a discounted return, and hence we have an effective
horizon $H=(1/(1-\gamma))\log (R_{max}/\epsilon)$ which guarantees
that the expected return in the first $H$ steps is within $\epsilon$
of the true return. For this reason we can consider the {\em
truncated discounted return} which is $E[\sum_{t=0}^H \gamma^t
r_t]$.

Using the generative model we can generate a random trajectory of
$\pi$ of length $H$. We start by applying the generative model on
$(s_0, \pi(s_0))$ and observing $(r_0,s_1)$. At time $t$ we input
the generative model $(s_t,\pi(s_t))$  observe $(r_t,s_{t+1})$, and
terminate at time $H$. We have generated a trajectory
$(s_0,\pi(s_0), r_0, s_1, \ldots, s_H,r_H,s_{H+1})$. The return of
the trajectory is $v=\sum_{t=0}^H r_t$, and clearly this is an
unbiased estimate of the true truncated discounted return, i.e.,
$E[v]=E[\sum_{t=1}^H \gamma^t r_t]$.


We can use the generative model in a similar way for a POMDP. The
main issue here is that the policy gets as inputs only the
observations and not the states. More specifically, we start by
applying the generative model on $(s_0, \pi(s_0))$  observing
$(r_0,o_1,s_1)$ and set the history to be $h_1=o_1$. At time $t$ we
input the generative model $(s_t,\pi(h_t))$  observe
$(r_t,o_{t+1},s_{t+1})$, and set $h_{t+1}=(h_t,o_{t+1})$. We
terminate at time $H$. The return of the trajectory is
$v=\sum_{t=0}^H \gamma^t r_t$, and clearly this is an unbiased
estimate of the true truncated discounted return, i.e.,
$E[v]=E[\sum_{t=1}^H \gamma^t r_t]$. Note that the policy gets only
the observations $o_t$ and not the states $s_t$.

We can use multiple estimates $v_i$ to get with high probability an
accurate estimate. Namely, set $m=(V_{max}^2/\epsilon^2)\log
(1/\delta)$, where $V_{max}=R_{max}/(1-\gamma)$. We run $m$
trajectories and for the $i$-th trajectory we have an estimate
$v_i$. Our final estimate is $\hat{V}^\pi(s_0) =(1/m)\sum_{i=1}^m
v_i$. This implies the following theorem:

\begin{theorem}
With probability $1-\delta$, for $m=O((V_{max}^2/\epsilon^2)\log
(1/\delta))$, we have that,
\[
|\hat{V}^\pi(s_0) -V^\pi(s_0)|\leq \epsilon
\]
\end{theorem}

\subsection{Trajectory tree}

Our goal is to build a single data structure, that latter, given
{\em any} policy $\pi$ we will be able to estimate $V^\pi(s_0)$. We
will first build such a data structure for a single unbiased
estimate, and then we will extend it for multiple estimates.

The data structure will be called {\em trajectory tree}, which will
be a tree of depth $H$. The nodes will be labeled by states, and the
edges by actions (and rewards). the process of creating a trajectory
tree would be stochastic using the generative model.

The trajectory tree will be a complete directed tree of degree $A$.
For each inner-node, the out edges are labeled by action $a\in A$
(each action labels one edge).
%
Initially, we label the root by $s_0$. We use the generative model
on input $(s_0,a)$, for each $a\in A$, and get $(s_{1,a},r_{1,a})$.
We label the $a$-th child by $s_{1,a}$ and the edge by $r_{1,a}$. We
continue the process, where given a node labeled by $s_t$, we use
the generative model on input $(s_t,a)$, for each $a\in A$, and get
$(s_{t+1,a},r_{t+1,a})$. We label the $a$-th child of $s_t$ by
$s_{t+1,a}$ and the edge by $r_{t+1,a}$.

Given a policy $\pi$ we can generate an unbiased trajectory, by
simply following the policy $\pi$ in the tree. Namely, we start at
the root, $s_0$, continue to the child $\pi(s_0)\in A$, call it
$s_1$, then continue to child $\pi(s_1)$, etc. The estimate for the
return of $\pi$ is the sum of the rewards on the path it follows,
i.e., $T(\pi)=\sum_{t=0}^H \gamma^t r_t$. Clearly, when we average
over the trajectory trees $T$ we have an unbiased estimate (since we
are generating the trajectories with the right probabilities). This
implies that $E_T [T(\pi)]=V^\pi(s_0)$. Note that if we use the same
tree $T$ for multiple polices $\pi_i$, we get an unbiased estimate
for each policy $\pi_i$, but the estimates of different policies
$\pi_i$ and $\pi_j$ are correlated (due to using the same tree $T$).

We can extend the trajectory tree to multiple trajectory trees, and
average the estimated returns. However, we will prefer to extend a
single tree, and let it encode multiple trees. More specifically,
given a parameter $m$, we create a complete tree of depth $H$ where
the degree of each inner node is $m|A|$, where we have $m$ edges for
each action $a\in A$. Otherwise we create the trajectory tree the
same way as before.

\subsection{Policy Evaluation: Finite hypothesis class}

Assume we have a finite hypothesis class $\Pi=\{\pi:S\rightarrow
A\}$. Our goal is to estimate for each $\pi\in \Pi$ its return and
select the policy with the highest estimated return.

For this task we build a multi-trajectory tree with the parameter
$m=(V^2_{max} /\epsilon^2)\log (|\Pi|/\delta)$. for each $\pi\in
\Pi$ we have $m$ estimates $v_i$ which we average
$\hat{V}^\pi(s_0)=(1/m)\sum_{i=1}^m v_i$. We are guaranteed that for
each $\pi\in \Pi$ we have that with probability $1-\delta/|\Pi|$
that $|\hat{V}^\pi(s_0)-V^\pi(s_0)|\leq \epsilon$. Using a union
bound over $\pi\in \Pi$ we have:

\begin{theorem}
With probability $1-\delta$, for $m=O((V^2_{max} /\epsilon^2)\log
(|\Pi|/\delta))$, we have that
\[
|V^{\hat{\pi}^*}(s_0)-V^{\pi^*}(s_0)|\leq \epsilon
\]
where $\hat{\pi}^*=\arg\max_{\pi\in\Pi} \hat{V}^\pi(s_0)$ and
$\pi^*=\arg\max_{\pi\in\Pi} V^\pi(s_0)$.
\end{theorem}
Note that the above theorem does not depend on the size of the state
space $|S|$ or the complexity of the optimal policy.

\subsection{Gradient based algorithm}

Assume that we have a parameterized policy class $\Pi$ where the
policies are smooth in the parameter. Namely we have policies
$\pi(\cdot;\theta)$ and the policy distribution is smooth in
$\theta$. Our goal is to compute the gradient of the expected
reward, and allow to improve the policy.

Fix a trajectory tree $T$. We want to compute $\nabla_\theta
T(\pi(\cdot;\theta))$, both efficiently and unbiased estimate.

For each node $i\in T$ we have a depth $d_i$ and history $h_i$
leading to it. The history $h_i$ has a reward $v^1_i$. Using policy
$\pi(\cdot;\theta)$ we induce a distribution over the leaves of $T$,
starting at node $i$, and let $v^2_{i,a}$ be the expected return
starting at node $i$ and performing action $a$.

Given the policy $\pi(\cdot;\theta)$ we have a  probability $P[i]$
of the trajectory reaching node $i$ when it follows policy
$\pi(\cdot;\theta)$, and a distribution over actions in node $i$
which is $p_{i,a}=\pi(a|i;\theta)$.

The expected value of the policy is
\[
E[T(\pi(\cdot;\theta))]=\sum_{t=0}^H \gamma^t
E[r_t|T,\pi(\cdot;\theta)]\propto \sum_{i\in T}\sum_{a\in A}
P[i](v^1_i+\gamma^{d_i}E[v^2_{i,a}]\pi(a|i,\theta)
\]
(Since we have that $\sum_{i\in T} P[i]=H$ the expressions are only
proportional.)

Using the chain rule we have,
\[
%\nabla_\theta
%
\frac{\partial}{\partial \theta} T(\pi(\cdot;\theta))= \sum_{i\in
T,a\in A} \frac{\partial T}{\partial p_{i,a}} \frac{\partial
p_{i,a}}{\partial \theta}
\]
This implies that
\[
\frac{\partial}{\partial \theta} T\propto \sum_{i\in T,a\in A}
P[i]\gamma^{d_i}E[v^2_{i,a}]\frac{\partial}{\partial \theta} p_{i,a}
\]
we are still left with the question of how do we get an unbiased
estimate.

One solution is to sample $i$ with probability proportional to
$P[i]\gamma^{d_i}$, and then sample, for each $a\in A$, a
continuation of a trajectory with return $V^2_{i,a}$ (and expected
return $v^2_{i,a}$). We can then return $V^2_{i,a}\nabla_\theta
p_{i,a}$, and our estimate would be $\sum_{a\in A}
\pi(a|i,\theta)V^2_{i,a} \nabla_\theta p_{i,a}$.

We can implement this sampling as follows. We select a depth $d$
from a geometric distribution with parameter $1-\gamma$ (i.e.,
$\Pr[d=\ell]\propto \gamma^\ell$). We follow policy $\pi(\cdot
;\theta)$ to depth $d$, and define the node we reached as node $i$.
For each action $a\in A$ we follow the trajectory from $i$ which
starts with action $a$, and uses policy $\pi(\cdot ;\theta)$ to
select the future actions. Let $V^2_{i,a}$ the return of this
trajectory (from node $i$ starting with action $a$). We return
$$\sum_{a\in A}
\pi(a|i,\theta)V^2_{i,a} \nabla_\theta p_{i,a}
$$

We described an efficient way to compute the gradient in an unbiased
way. This implies that if we follow the gradients we will reach a
local maximum of the expected return.

For the complexity analysis, note that we do not need to compute the
entire trajectory tree in advance, we can do it in a lazy way. We
add a path for each gradient we compute. This implies that for each
gradient estimation we have $O(H|A|)$ call to the generative model.

\section{Near Optimal Policy}

We would like to define a near optimal policy given a generative
model. The idea is that we define an algorithm, that given a state
$s$, runs and queries the generative model as a black box, and
return an action $a$. In this way the algorithm defines implicitly a
policy. Our goal is that the running time of the algorithm would be
efficient. Our algorithm would run in time exponential in $H$, the
effective horizon, but independent of the state space size $|S|$.

Our main goal is to establish the following theorem:

\begin{theorem}
There exists an algorithm that,  given access to a generative model,
defines a policy $\pi$ such that for any state $s$ we have
\[
V^*(s)-V^\pi(s)\leq \epsilon
\]
and the running time of the algorithm is exponential in $H$, the
effective discounted horizon, but independent of the state space
size.
\end{theorem}

We start by looking at a restricted case. Assume that for any state
$s\in S$ and action $a\in A$ the support of the distribution of the
next state $p(\cdot|s,a)$ is bounded by $d$. (We will latter get rid
of this assumption.)

Fix a state $s$ (the input to the algorithm). Using the generative
model, recover all the states that can be reached from $s$ with at
most $H$ steps. (Assume that we can guarantee to recover all those
states, actually, we should have a small error probability, for low
probability states.)

Let $\Delta(s,H)$ be the set of states reachable from $s$ with at
most $H$ steps. Given our assumption on the support we have
$|\Delta(s,H)|\leq (d|A|)^H $. We restrict the MDP to the set
$\Delta(s,H)$ and compute the optimal $Q$-function on this MDP,
denoted by $Q^{*,H}(s,a)$. Let $a'$ be the action that maximizes the
$Q^{*,H}(s,\cdot)$. We have that
\[
Q^*(s,a)-\epsilon \leq Q^{*,H}(s,a)\leq Q^*(s,a)
\]
where the lower bound is due to the horizon definition.

We would like to show that if in each step we select an action that
nearly maximizes the $Q^*$-function, then the policy is near
optimal.

\begin{lemma}
If for every state $s$, with probability $1-\delta$, we have
\[
Q^*(s,a^*)-Q^*(s,\pi(s))\leq \epsilon
\]
where $a^*$ is the optimal action in state $s$. Then, for every
state $s$ we have
\[
V^*(s)-V^\pi(s)\leq \frac{\epsilon+2\delta V_{max}}{1-\gamma}
\]
\end{lemma}

\begin{proof}
\begin{align*}
E[Q^*(s,\pi(s))] &\geq (1-\delta)(Q^*(s,a^*)-\epsilon)-\delta
V_{max}\\
&\geq Q^*(s,a^*)-\epsilon -2\delta V_{max}
\end{align*}
Let $\beta=\epsilon+2\delta V_{max}$. The following claim completes
the proof.
\begin{claim}
If for every state $s$ we have
\[
Q^*(s,a^*)-Q^*(s,\pi(s))\leq \beta
\]
Then
\[
V^*(s)-V^\pi(s)\leq \frac{\beta}{1-\gamma}
\]
\end{claim}

\begin{proof}
Let $\pi_i$ be the policy that runs $\pi$ in the first $i$ steps,
and then switches to $\pi^*$. For $\pi_1$ we have
$V^*(s)-V^{\pi_1}(s)\leq \beta$. For $\pi_i$, since the first $i$
steps are identical, we have $V^{\pi_i}(s)-V^{\pi_{i+1}}(s)\leq
\gamma^i\beta$. By summing the inequalities we have that
\[
V^*(s)-V^{\pi_H}(s)\leq \sum_{t=0}^H
\gamma^t \beta\leq \frac{\beta}{1-\gamma}
\]
\end{proof}
\end{proof}

We now would like to address the general case. For example, consider
the case that the next state distribution is uniform over $S$. We
clearly cannot hope to approximate this model well, simply too
large.

Our main insight will be to use the trajectory trees, to approximate
the neighborhood around a state $s$, where the root is $s$. The main
issue will be to understand the errors in the process.

The basic idea is given a state $s$, we sample each action $a\in A$
for $m$ times. Let $s'_{a,i}$ be the next state in the $i$-th time
we execute action $a$ from state $s$. We now would like to compute
recursively the return from each $s'_{a,i}$.

The main conceptual issue is whether we are making progress. We
wanted to estimate $V^*(s)$, and we have now to estimate
$V^*(s'_{a,i})$. It seems that instead to estimate the optimal value
from one state we need to approximate it from $m|A|$ state. The main
point is that due to discounting we are making progress. We can
allow now a larger error rate!

First, we consider the error due to the sampling of $m$ state.
\begin{lemma}
\label{lem:finite-smaple} Let $\hat{Q}^*(s,a)=r(s,a)+(\gamma
/m)\sum_{i=1}^m V^*(s'_{a,i})$, where $s'_{a,i}\sim p(\cdot|s,a)$.
With probability $1-e^{-\epsilon^2 m/V^2_{max}}$ we have
$|Q^*(s,a)-\hat{Q}^*(s,a)|\leq \epsilon$.
\end{lemma}
The proof follows from an application of the Chernoff concentration
bounds.

\paragraph{Near optimal algorithm}\ \\

\bigskip
\noindent
%{\tt EstimateQ(0,s,a):  return 0} \\
%\smallskip
{\tt EstimateQ(n,s,a)}:\\
%\smallskip
IF $n=0$ THEN return $0$\\
ELSE\\
Sample $m$ times action $a$ in state $s$ (using the generative
model).\\
Let $s'_{a,i}$ be the next next state in the $i$-th sample.\\
Return $(1/m)\sum_{i=1}^m$ {\tt EstimateV(n-1,s,a)}\\

\medskip
\noindent
{\tt EstimateV(n,s):} return $\max_a$ {\tt EstimateQ(n,s,a)}.\\

\medskip
\noindent
{\tt main(s)}: return {\tt EstimateV(n,s)}\\
\bigskip
%For each $a\in A$ compute $q_a=${\tt EstimateQ(H,s,a)$

For the algorithm, we have two sources of errors. The first is due
to the finite sample $m$, and is addressed in
Lemma~\ref{lem:finite-smaple}. The second, and more challenging, is
the recursive calls. We will use the discounting to show that the
error decrease.

We start with a few notations:
\begin{align*}
Q^n(s,a)&={\tt EstimateQ}(n,s,a)\\
V^n(s)&={\tt EstimateV}(n,s)
\end{align*}
The computation is simply
\begin{align*}
Q^n(s,a)&=r(s,a)+(\gamma/m)\sum_{i=1}^m V^{n-1}(s'_{a,i})\\
V^{n-1}(s)&=\max_a Q^{n-1}(s,a)
\end{align*}
We define the error bounds as
\begin{align*}
|Q^*(s,a)-Q^n(s,a)|&\leq \alpha_n\\
|V^*(s)-V^n(s)|&\leq \alpha_n
\end{align*}
where the first inequality implies the second.

Trivially, we have $\alpha_0=V_{max}$. For the error bound we have
\begin{align*}
|Q^*(s,a)-Q^n(s,a)|=&\gamma|E[V^*(s')]-(1/m)\sum_{i=1}^m
V^{n-1}(s'_{a,i})|\\
\leq &\gamma|E[V^*(s')]-(1/m)\sum_{i=1}^m V^{*}(s'_{a,i})|\\
&+\gamma|(1/m)\sum_{i=1}^m V^{*}(s'_{a,i})-(1/m)\sum_{i=1}^m
V^{n-1}(s'_{a,i})|
\end{align*}
This implies that $\alpha_n \leq \gamma(\epsilon+\alpha_{n-1})$,
which implies that
\[
\alpha_H\leq \sum_{i=0}^H \gamma^i \epsilon + \gamma^H V_{max}\leq
\frac{\epsilon}{1-\gamma}+\gamma^H V_{max}
\]

To complete the algorithm, we set the parameter $\epsilon'$, by
setting $\epsilon=\epsilon'(1-\gamma)$ and
$H=(1/(1-\gamma))\log(V_{max}/\epsilon')$. This implies that
$\alpha_H\leq 2\epsilon'$ and have $V^*(s)-V^\pi(s)\leq 2\epsilon'$.

\paragraph{Lower bound}\ \\

The exponential dependence on the horizon $H$ is inherent. To see
this, consider an MDP which is a complete $|A|$-degree tree of depth
$H$, where for each node each outgoing edge is associated with a
different $a\in A$. All the rewards are zero, except one leaf where
the reward is one, which is selected at random.

Any algorithm would have to check, on average, at least half the
leaves before it finds the unique leaf.

\begin{theorem}
Any algorithm that uses a generative model need to make
$\Omega(|A|^H)$ queries.
\end{theorem}




\section{Inverse Reinforcement Learning}

The classical reinforcement problem has as inputs the dynamics and
the rewards and the output is the optimal policy. In the inverse
reinforcement learning problem we are given access to the optimal
policy and the dynamics, and would like to learn a reward function
which induces the given optimal policy.

More specifically, we are given as inputs the states $S$, actions
$A$, dynamics $p(\cdot|s,a)$, and access to the optimal policy
$\pi^*$, either explicitly or implicitly, by observing trajectories
of $\pi^*$.

We have a few related problems:
\begin{enumerate}
\item
Behavioral cloning: given trajectories learn the optimal policy.
\item
Inverse Reinforcement Learning (IRL): which recovers a reward
function $R$ for which $\pi^*$ is optimal.
\item
Apprentice learning: Learn a near-optimal policy given access to
trajectories of an optimal policy.
\end{enumerate}

\subsection{Behavioral Cloning}

Formulated as a classical supervised classification problem. Given
trajectories generated from $\pi^*$, we extract pairs $(s_t,a_t)$,
where $\pi^*(s_t)=a_t$. We do not need to know the dynamics or the
rewards. We define a classification problem, where the training
examples are ${\cal S}=\{(s_t,a_t)\}$.

We can now learn a classifier for ${\cal S}$, and use any
classifier, DNN, SVM, decision tree, and more. We learn the optimal
policy as a classification problem, and if we have low error we can
hope for performance near that of the optimal policy.

There are a few problems with the approach. After we make errors we
might reach completely new states, which we never encountered before
(since the optimal policy never reaches them). Also, the different
errors might have very different effects, but we treat them all
equally. Other issues include generalization, which would depend on
the test and train distributions (of states) being identical, but
small differences in the policies might lead to significant
differences in the distributions.


\subsection{IRL: small state space}

Assume we can use a look-up table to represent the policies. We are
given the set of states $S$, actions $A$, $p(\cdot|s,a)$ and the
optimal policy $\pi^*$. (All are given explicitly, as look-up
tables.) The output of the algorithm is a reward function $r(s,a)$
for which $\pi^*$ is optimal.

We first characterize all the rewards $R$ for which $\pi^*$ is
optimal. Bellman optimality requires that for every $s\in S$ and
$a\in A$ we have,
\[
Q^*(s,\pi^*(s))\geq Q^*(s,a)
\]
First we will try to get a more explicit characterization.

For simplicity of the notation let $a_1=\pi^*(s)$ for every $s\in
S$. (This is simply renaming the actions.) Also, we associate the rewards with states (rater
then state-action pairs). Define matrices $P^a$,
for $a\in A$, as follows: $P^a[i,j]=p(s_j|s_i,a)$, which is the
transition probabilities using action $a$. Since $a_1$ is the
optimal action, we define $P^*=P^{a_1}$.

For the optimal value function $V^*\in \mathbb{R}^{|S|}$ has
\[
V^*=R+\gamma P^*V^*
\]
This implies that $V^*=(I-\gamma P^*)^{-1}R$. (The inverse exists,
since the eigenvalues of $P^*$ are in the unit circle, hence, all
the eigenvalues of $I-\gamma P^*$ are non-zero.)

The optimal state-action function is,
\[
Q^*=R+\gamma P^a V^*=R+\gamma P^a (I-\gamma P^*)^{-1}R
\]
From the Bellman optimality we have $Q^*(s,\pi^*(s))- Q^*(s,a)\geq
0$ which implies,
\[
(P^*-P^a) (I-\gamma P^*)^{-1}R\geq 0
\]

This establishes the following claim:
\begin{claim}
Reward function $R$ is optimal iff $(P^*-P^a) (I-\gamma
P^*)^{-1}R\geq 0$.
\end{claim}

Although we have an exact characterization, we have a few
challenges.
\begin{enumerate}
\item
Trivial solutions: We can simply set $R=0$ and any policy is
optimal!
\item
What happens if we observe only trajectories of $\pi^*$ and not the
explicit policy and dynamics.
\item
Large state space.
\end{enumerate}

To overcome the ambiguity of the reward function we can set an
objective that will force a unique outcome. One solution is a {\em
max margin}, specifically,
\[
\max_{s\in S} \left\{ Q^*(s,a^*)-\max_{a\neq a^*} Q^*(s,a) \right\}
\]
The resulting linear program is,\\
\begin{align*}
&\max \lambda\\
\forall a^*\neq a\;\;\;\; & (P^*-P^a) (I-\gamma P^*)^{-1}R\geq \lambda  \\
&-R_{max}\leq R\leq R_{max}
\end{align*}


\subsection{AL: Linear function approximation}

In this section we consider the case where we are given access to
trajectories of the optimal policy, in a large state space. Our goal
is to find a near optimal policy.
%Since we have a large state space we would use function approximation.
%
We consider a  simple case that the rewards are linear in the state
representation. For each state $s\in S$ we have
$\phi(s)\in\mathbb{R}^n$. There is a reward weights
$w\in\mathbb{R}^n$. The reward in state $s$ is $w^\top \phi(s)$.

Consider the value function of the policy $\pi$.
\begin{align*}
V^*(s_0) =& E[\sum_{t=0}^\infty \gamma^t r_t|\pi]\\
 =& E[\sum_{t=0}^\infty \gamma^t w^\top \phi(s_t)|\pi]\\
 =& w^\top  \sum_{t=0}^\infty \gamma^tE[ \phi(s_t)|\pi]\\
 =& w^\top \mu(\pi)
\end{align*}
where $\mu(\pi)=\sum_{t=0}^\infty \gamma^tE[ \phi(s_t)|\pi]$. Note
that $\mu(\pi)$ depends on the dynamics (steady state distribution)
that policy $\pi$ induces.

For IRL we like to recover a weight vector $w$ such that for any
policy $\pi$ we have,
\[
w^\top \mu(\pi^*)\geq w^\top \mu(\pi)
\]
First, we can approximate $\mu(\pi)$ from sample of $m$
trajectories,
\[
\mu(\pi)\approx \hat{\mu}(\pi)=\frac{1}{m}\sum_{i=1}^m \sum_{t=0}^H
\gamma^t \phi(s_t)
\]
Second, observe that if we have $\|\mu(\pi)-\hat{\mu}(\pi)\|_2\leq
\epsilon$ then for any $w$ we have
\[
|w^\top \mu(\pi)-w^\top\hat{\mu}(\pi)|\leq \epsilon \|w\|_2
\]

For the AL we like to find a near optimal policy $\pi'$, but we do
not have access to the rewards. However, if we find a policy $\pi'$
such that $\mu(\pi')\approx\mu(\pi^*)$, then for any weights $w$ the
policy $\pi'$ will give a similar expected return as $\pi^*$.

Consider first the IRL problem. As before, we have a problem that
$w=0$ is a solution. We will overcome this by defining a max margin
linear program.
\begin{align*}
& \max \lambda\\
\forall \pi\;\;\;\;  &  w^\top \mu(\pi^*) \geq w^\top\mu(\pi) +\lambda \\
& \|w\|_2^2 \leq 1
\end{align*}

A clear weakness is that we have an exponential size linear program.
We will overcome this by an {\em incremental constraint generation}
which will solve the AL. We maintain a set of policies $\Pi$, where
initially $\Pi=\emptyset$. Initially we select some $\pi^0$ and let
$\mu^0=\mu(\pi^0)$. At iteration $i$ we solve
\[
\max_w \min_{\pi_j\in \Pi} w^\top (\mu^*-\mu^j)
\]
If the value is at most $\epsilon$ we terminate. Otherwise let $w^i$
be the maximizer. We solve for the optimal policy $\pi^i$ given
weights $w^i$ and add $\pi^i$ to $\Pi$. At termination, we compute a
$\mu=\sum_{i=1}^k a_i\mu^i$ which minimizes $\|\mu^*-\mu\|_2$, where
$a_i\geq 0$ and $\sum_i a_i=1$.

The correctness of the constraint generation follows from the
following lemma.
\begin{lemma}
At termination $\|\mu^*-\mu\|_2\leq \epsilon$
\end{lemma}
We can implement $\mu$ by selecting policy $\pi^i\in\Pi$  with
probability $a_i$ and running it to completion.

The main issue is the rate of convergence. How long will it take to
terminate. The following lemma guarantees the fast convergence.
\begin{lemma}
The number of iterations is bounded by
\[
O(\frac{n}{(1-\gamma)^2\epsilon^2}\log\frac{n}{(1-\gamma)\epsilon})
\]
\end{lemma}
\section{Bibliography Remarks}

The trajectory trees for approximate planning and evaluation is
taken from \cite{KearnsMN99,KearnsMN02}.

The inverse reinforcement learning follows \cite{PieterN04,NgR00}


%\bibliography{../bib-lecture}
%\bibliographystyle{plain}
