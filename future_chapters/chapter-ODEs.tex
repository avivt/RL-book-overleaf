Ordinary differential equations (ODEs) are fundamental tools in mathematical modeling, used to describe dynamics and processes in various scientific fields. This chapter provides an introduction to ODEs, with a focus on linear systems of ODEs, their solutions, and stability analysis.


\section{Definitions and Fundamental Results}
An ordinary differential equation (ODE) is an equation that involves a function of one independent variable and its derivatives. The most general form of an ODE can be expressed as:
\[
F(x, y, y', y'', \dots, y^{(n)}) = 0
\]
where $ y = y(x) $ is the unknown function, and $ y', y'', \dots, y^{(n)} $ represent the first through $ n $-th derivatives of $ y $ with respect to $ x $.

A linear ODE has the form:
\[
a_n(x) y^{(n)} + a_{n-1}(x) y^{(n-1)} + \cdots + a_1(x) y' + a_0(x) y = g(x),
\]
where $ a_0, a_1, \dots, a_n $ and $ g $ are continuous functions on a given interval.

Typically, given an ODE, the goal is to find a set of functions $y$ that solve it. We may also be interested in other properties of the set of solutions, such as their limits.

\begin{example}    
Consider the first-order linear ODE
\[
y' = ay + b,
\]
where $ a $ and $ b $ are constants. This equation can be solved using an integrating factor. The integrating factor, $ \mu(x) $, is given by $ \mu(x) = e^{-ax} $. Multiplying through by this integrating factor, the equation becomes:
\[
e^{-ax} y' = ae^{-ax} y + be^{-ax}.
\]
This simplifies to:
\[
(e^{-ax} y)' = be^{-ax}.
\]
Integrating both sides with respect to $ x $ gives:
\[
e^{-ax} y = -\frac{b}{a} e^{-ax} + C,
\]
where $ C $ is the constant of integration. Solving for $ y $, we obtain:
\[
y(x) = -\frac{b}{a} + Ce^{ax}.
\]
Note that if $a<0$, we have that $\lim_{x \to \infty} y(x) = -\frac{b}{a}$ for all the solutions of the ODE.
\end{example} 

A fundamental result in ODE theory is the Picard-Lindelöf theorem, also known as the existence and uniqueness theorem.
\begin{theorem}[Existence and Uniqueness]\label{thm:ODE_picard}
Consider the ODE given by
\[
y'(x) = f(x, y(x)), \quad y(x_0) = y_0,
\]
where $f: [a, b] \times \mathbb{R} \rightarrow \mathbb{R}$ is a function. Assume that $f$ satisfies the following conditions: (1) $ f $ is \textbf{continuous} on the domain $ [a, b] \times \mathbb{R} $.
(2) $ f $ satisfies a \textbf{Lipschitz condition} with respect to $ y $ in the domain, i.e., there exists a constant $ L > 0 $ such that
$ 
   |f(x, y_1) - f(x, y_2)| \leq L |y_1 - y_2|
$
   for all $ x \in [a, b] $ and $ y_1, y_2 \in \mathbb{R} $.
Then, there exists a \textbf{unique} function $ y: [a, b] \rightarrow \mathbb{R} $ that solves the ODE on some interval containing $ x_0 $.
\end{theorem}
The proof of this theorem involves constructing a sequence of approximate solutions using the method of successive approximations and showing that this sequence converges to the actual solution of the differential equation.
For a detailed proof and further exploration of this theorem, refer to classic texts on differential equations such as \cite{hirsch2013differential}.

\subsection{Systems of Linear Differential Equations}
When dealing with multiple interdependent variables, we can extend the concept of linear ODEs to systems of equations. These are particularly useful in modeling multiple phenomena that influence each other.

Consider a system of linear differential equations represented in matrix form as follows:

\begin{equation}\label{eq:lin_ODE}    
\mathbf{y}' = A \mathbf{y} + \mathbf{b},
\end{equation}
where $ \mathbf{y} $ is a vector of unknown functions, $ A $ is a matrix of coefficients, and $ \mathbf{b} $ is a vector of constants. This compact form encapsulates a system where each derivative of the component functions in $ \mathbf{y} $ depends linearly on all other functions in $ \mathbf{y} $ and possibly some external inputs $ \mathbf{b} $. We shall now present the general solution to the ODE \eqref{eq:lin_ODE}.

Let us first define the matrix exponential.

\begin{definition}
The matrix exponential, $ e^{Ax} $, where $ A $ is a matrix, is defined similarly to the scalar exponential function but extended to matrices,    
\[
e^{Ax} = \sum_{k=0}^{\infty} \frac{x^k A^k}{k!}.
\]
\end{definition}
The matrix exponential is fundamental in systems theory and control engineering, as it provides a straightforward method for solving linear systems of differential equations.

\begin{proposition}
    The solutions of the system of linear differential equations in \eqref{eq:lin_ODE} are given by $ \mathbf{y}(x) = e^{Ax} \mathbf{y}_0 + \mathbf{y}_p$, where $\mathbf{y}_p$ is such that $A\mathbf{y}_p = -\mathbf{b}$.
\end{proposition}

\begin{proof}
    Let us first consider the homogeneous case $ \mathbf{b} = 0 $. To prove that $ \mathbf{y}(x) = e^{Ax} \mathbf{y}_0 $ solves $ \mathbf{y}' = A \mathbf{y} $, we differentiate $ \mathbf{y}(x) $ with respect to $ x $ and show that it satisfies the differential equation $\mathbf{y}' = A \mathbf{y}$.

    The derivative of $ \mathbf{y}(x) $ with respect to $ x $ is given by:
   \[
   \frac{d}{dx} \mathbf{y}(x) = \frac{d}{dx} \left(e^{Ax} \mathbf{y}_0\right).
   \]
   Applying the derivative to the series expansion of $ e^{Ax} $, we get:
   \[
   \frac{d}{dx} \left(e^{Ax}\right) = \frac{d}{dx} \left( \sum_{k=0}^{\infty} \frac{(Ax)^k}{k!} \right) = \sum_{k=0}^{\infty} \frac{d}{dx} \left( \frac{(Ax)^k}{k!} \right).
   \]
   Using the power rule and the properties of matrix multiplication, we find:
   \[
   \sum_{k=1}^{\infty} \frac{A (Ax)^{k-1} k}{k!} = A \sum_{k=1}^{\infty} \frac{(Ax)^{k-1}}{(k-1)!} = A e^{Ax}.
   \]
   Therefore,
   \[
   \frac{d}{dx} \mathbf{y}(x) = A e^{Ax} \mathbf{y}_0.
   \]
Substituting $ \mathbf{y}(x) $ back into the original differential equation:
   \[
   \mathbf{y}' = A \mathbf{y} \quad \Rightarrow \quad A e^{Ax} \mathbf{y}_0 = A \mathbf{y}.
   \]
   Since $ \mathbf{y}(x) = e^{Ax} \mathbf{y}_0 $, it follows that:
   \[
   A e^{Ax} \mathbf{y}_0 = A \mathbf{y}(x).
   \]
To show that $e^{Ax} \mathbf{y}_0$ is the only possible solution, note that at $x=0$, $e^{Ax} \mathbf{y}_0 = \mathbf{y}_0$. Therefore, for any initial condition, we have found a solution, and the uniqueness follows from Theorem \ref{thm:ODE_picard}.

Now, for the case $\mathbf{b} \neq 0 $, let $\mathbf{y}_p$ such that $A \mathbf{y}_p = -\mathbf{b}$. We have that for $ \mathbf{y}(x) = e^{Ax} \mathbf{y}_0 + \mathbf{y}_p$, 
\begin{equation*}
\mathbf{y}'(x) = A e^{Ax} \mathbf{y}_0 = A e^{Ax} \mathbf{y}_0 + A\mathbf{y}_p - A\mathbf{y}_p = A \mathbf{y}(x) + \mathbf{b}.
\end{equation*}
\end{proof}


\section{Asymptotic Stability}

We will be interested in the asymptotic behavior of ODE solutions. In particular, we shall be interested in the following stability definitions.

\begin{definition}[Stability]
A solution \( y(x) \) of a differential equation is called \textbf{stable} if, for every \( \epsilon > 0 \), there exists a \( \delta > 0 \) such that for any other solution \( \tilde{y}(x) \) with \( |\tilde{y}(0) - y(0)| < \delta \), it holds that \( |\tilde{y}(x) - y(x)| < \epsilon \) for all \( x \geq 0 \).    
\end{definition}
\begin{definition}[Asymptotic Stability]
A solution \( y(x) \) of a differential equation is called \textbf{globally asymptotically stable} if for any other solution \( \tilde{y}(x) \), we have  \( \lim_{x \to \infty} |\tilde{y}(x) - y(x)| = 0 \).
\end{definition}

Intuitively, asymptotic stability means that not only do perturbations remain small, but they also decay to zero as $x$ progresses, causing the perturbed solutions to eventually converge to the stable solution. 

\begin{definition}[Global Asymptotic Stability]
A solution \( y(x) \) of a differential equation is called \textbf{asymptotically stable} if it is stable and additionally, there exists a \( \delta > 0 \) such that if \( |\tilde{y}(0) - y(0)| < \delta \), then \( \lim_{x \to \infty} |\tilde{y}(x) - y(x)| = 0 \).
\end{definition}

We have the following result for the system of linear differential equations in \eqref{eq:lin_ODE}.

\begin{theorem}
Consider the ODE in \eqref{eq:lin_ODE}, and let $A\in \mathbb{R}^{N \times N}$ be diagonizable. If all the eigenvalues of $A$ have a negative real part, and let $\mathbf{y}_p$ such that $A\mathbf{y}_p = -\mathbf{b}$, then $y = \mathbf{y}_p$ is a globally asymptotically stable solution.
\end{theorem}

\begin{proof}
    We have already established that every solution is of the form $ \mathbf{y}(x) = e^{Ax} \mathbf{y}_0 + \mathbf{y}_p$. Let $\lambda_i, v_i$ denote the eigenvalues and eigenvectors of $A$. Since $A$ is diagonizable, we can write $A y_0 = \sum_{i=1}^N \lambda_i v_i^T y_0$, so
    \[
e^{Ax}y_0 = \sum_{k=0}^{\infty} \frac{x^k A^k y_0}{k!} = \sum_{k=0}^{\infty}\sum_{i=1}^N \frac{x^k \lambda_i^k v_i^T y_0}{k!} = \sum_{i=1}^N e^{\lambda_i x}v_i^T y_0.
\]
If $\lambda_i$ has a negative real part, then $\lim_{x \to \infty} e^{\lambda_i x} = 0$. Thus, if all the eigenvalues of $A$ have a negative real part, $\lim_{x \to \infty} e^{Ax}y_0 = 0$ for all $y_0$, and the claim follows.
\end{proof}

A similar result can be shown to hold for general (not necessarily diagonizable) matrices. We state here a general theorem (see, e.g., Theorem 4.5 in \cite{khalil2002nonlinear}) without proof.

\begin{theorem}
Consider the ODE $\mathbf{y}' = A \mathbf{y}$, where $A\in \mathbb{R}^{N \times N}$. The solution $y = 0$ is globally asymptotically stable if and only if all the eigenvalues of $A$ have a negative real part.
\end{theorem}
